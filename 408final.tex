\documentclass[12pt]{article}
\usepackage[utf8]{inputenc}
\usepackage{amsmath}
\usepackage{mathtools}
\usepackage{amssymb}
\usepackage{graphicx}
\usepackage{enumerate}
\usepackage{enumitem}
\usepackage{verbatim}
\usepackage{indentfirst}
\usepackage[hidelinks]{hyperref} %no boxes around links
\usepackage{xcolor}
\usepackage{alltt}
\usepackage{textcomp}
\usepackage{slashed}
\usepackage[margin=0.6in]{geometry}
\usepackage{esvect}
\usepackage{titlesec}
\usepackage{braket}
\usepackage{tensor}
\usepackage{cancel}
\usepackage{color}
\usepackage{wrapfig}
\usepackage{subfig}
\usepackage{float}
\usepackage[figurename=]{caption} %allows to write labeless-figure number captions
\usepackage{sidecap}
\usepackage{graphics}
\usepackage{multicol}
\usepackage{lipsum}
\usepackage{setspace}


    %tikz packages
\usepackage{tikz}
\usepackage{pgfplots}
\usetikzlibrary{pgfplots.polar}
\usetikzlibrary{decorations.markings} 
% \usepackage{tikz-feynman}

    %write all math in ds
\everymath{\displaystyle}
    %allow pagebreaks during displaystyle
\allowdisplaybreaks


    %define new commands
\newcommand{\declarecommand}[1]{\providecommand{#1}{}\renewcommand{#1}}
\declarecommand{\ds}{\displaystyle}
\declarecommand{\nd}{\noindent}
\declarecommand{\phi}{\varphi}
    %\declarecommand{\epsilon}{\varepsilon}     made some changes to these
    \declarecommand{\ve}{\varepsilon}   %       for levi-civita use
    \declarecommand{\e}{\epsilon}
\declarecommand{\R}{\mathbb{R}}
\declarecommand{\del}{\partial}
\declarecommand{\d}{\delta}
\declarecommand{\l}{\ell}
\declarecommand{\L}{\mathcal{L}}
\declarecommand{\J}{\mathcal{J}}
\declarecommand{\tr}{\text{tr}}
\renewcommand{\t}{\text}
\declarecommand{\1}{\mathbb{1}}
\DeclareMathOperator{\sech}{sech}
\declarecommand{\A}{\mathcal{A}}


\titleformat{\section}{\large\scshape\raggedright}{}{0em}{} % Section formatting



    %tag form for hyperrefs
\newtagform{blue}{\color{blue}(}{)}




%fancy r
\usepackage{calligra}
\DeclareMathAlphabet{\mathcalligra}{T1}{calligra}{m}{n}
\DeclareFontShape{T1}{calligra}{m}{n}{<->s*[2.2]callig15}{}

\newcommand{\scripty}[1]{\ensuremath{\mathcalligra{#1}}}

\titleformat{\section}{\large\scshape\raggedright}{}{0em}{} % Section formatting



\begin{document}

\begin{center}
    \Large \fontfamily{qtm}  \textbf{A Time Series Analysis of the Linear\\  Chirp Rate of Binary Black Hole Merge Signals}\\
    \vspace{5pt} 
    \normalsize Jace Alloway, 1006940802\, /\, alloway1\\ PHY408 Final Report: Due Wednesday, April 24, 2024

    \vspace{20pt}
\begin{minipage}{40em}
    \textbf{Abstract} \hspace{10pt} This work outlines the time series analysis of simulated binary black hole (BBH) merge signals in the form of gravitational displacement waves with incorporated aLIGO noise. The purpose of this analysis was to extract time-dependent frequency information from wave signals using a Linear Chirp Transform (LCT) function. The data file used was a simulated 6-second merge signal between two black holes of mass $102.0\, M_S$ and $38.0\, M_S$ of z-spins $0.672$ and $0.612$, respectively, with a PSNR of $5.726$. Noise was filtered from the waveform and the post-merge tail was trimmed, after which the signal was windowed and broadcast into a two-dimensional chirp-frequency domain. The chirp rate and base frequency were found to be $\beta = 1.81\,s^{-2}$ and $\omega = 7.92\, s^{-1}$, respectively, and the frequency evolution of the signal was plotted. Limitations, difficulties, and future improvements of the analysis were discussed. 
\end{minipage}
\vspace{20pt}
\end{center}

\nd \hrulefill  

\vspace{10pt}



\begin{multicols}{2}
    
\begin{center}
    \textbf{I \hspace{5pt} INTRODUCTION}

\vspace{5pt}

    \textit{A. Background}

\vspace{-5pt}
\end{center}

The analysis of gravitational wave disturbances have become a recent interest among physicists since the first detection of gravitational waves at the Laser Interferometer Gravitational-Wave Observatory (LIGO) in 2015. 
These waves are typically generated by the merging of binary black hole (BBH) systems of various masses and spins and are measured by interferometer laser fluctuations caused by the distortion of spacetime. These measurements can be converted into time-dependent displacement data in the form of a wave and further analyzed to extract information about the bodies involved and the processes leading up to the merge$^{[2]}$. 

This paper discusses the time series analysis process of extracting chirp rate and frequency evolution information from a provided BBH merger signal using filters and two-dimensional Fourier Transform (FT) methods. Further methods such as the introduction of time-dependent window sinusoids were also tested in an attempt to extract instantaneous frequency-time dependence of the signal, however it was obvious throughout the duration of the analysis that it could not be possible due to the density of the dataframe. 

The data selected was a list of simulated 6$s$ BBH mergers from 410Mpc with advanced LIGO (aLIGO) noise added to the waveform to mimick true interferometer measurements$^{[1]}$. 
\begin{center}
    \textit{B. Theory}\\
\end{center}

A `chirp' is a signal whose frequencies are changing with time, from which the rate is the change in frequency with respect to the change in time $\frac{df}{dt}$. A linear chirp is a signal whose frequencies change linearly with time, such as$^{[3]}$ 
\[
    Ae^{-2\pi i (\omega t + \beta t^2 + \phi)},\tag{1}
\] 
where $A$ is an amplitude, $\omega$ is the base frequency ($s^{[-1]}$), $\beta$ is the chirp rate ($s^{-2}$) and $\phi$ is a phase. The frequency rate of change is then $\omega + 2\beta t$, which is linear in time. There also exists non-linear chirp dependences, but these were not investigated as all BBH signals were assumed to have a linear dependence on $\beta$ due to an absence of nonlinearities in short merger signals. 

The extraction of the chirp rate $\beta$ and base frequency $\omega$ can be calculated using a modified FT algorithm, called a Linear Chirp Transform (LTC)$^{[3]}$, where the input signal $f(t)$ is projected along a frequency axis and a chirp axis, 
\[
    F_L(\omega, \beta) = \int_{-\infty}^\infty\, dt\, f(t)e^{-2\pi i (\omega t + \beta t^2)} \tag{2}  
\]
from which the maxima of the output signal is located at the values of $\omega$ and $\beta$ of the input signal. This transformation can be discretized into Python by using the Convolution Theorem to write 
\begin{align*}
    F_L(\omega, \beta) &= \int_{-\infty}^\infty\, dt\, [f(t)e^{-2\pi i\beta t^2}] \, e^{-2\pi i \omega t}\tag{3}\\
    &=\mathcal{F}\{f(t)e^{-2\pi i\beta t^2}\}\tag{4}\\
    &=\mathcal{F}\{f(t)\}\ast \mathcal{F}\{e^{-2\pi i\beta t^2}\}\tag{5}
\end{align*}
which produces a two-dimensional array in $\omega$ and $\beta$, where $\mathcal{F}$ is defined as the FT operator. To minimize spectral leakage and noise effects, it is also useful to window and to isolate the input signal along the time interval prior to the completion of the merge$^{[4]}$.  

For nonlinear chirp signals, one can define a Joint-Chirp-Rate-Time-Frequency Transformation (JCTFT)$^{[2]}$ which allows for an explicit analysis of the time-dependence of the signals chirp rate and frequency. This can be done by defining a complex window function 
\[
    y_C(\beta, t - \tau, \omega) = N e^{-(t - \tau)^2(\omega^2/2 + 2\pi i \beta)} \tag{6}   
\]
\nd where $N$ is a normalization and $\tau$ is a shifted time axis. This window function isolates specific rates along the input signal as a function of time, which can again be projected along the frequency axis with a FT
\[
    F_{J}(\omega, \tau, \beta) = \int_{-\infty}^\infty dt\, f(t)y_C(\beta, t - \tau, \omega) e^{-2\pi i \omega t}.\tag{7} 
\]
This can be discretized into Python using the same FT, Convolution Theorem method as in (5). 

This analysis discusses both methods of extracting chirp rate information from the input signal, however due to the density of data points and a small sampling rate, only results from a LCT (2) could be produced. 




\vspace{10pt}

\begin{center}
    \textbf{II \hspace{5pt} ANALYSIS AND RESULTS}

\vspace{5pt}

    \textit{A. Importing Data}

\vspace{-5pt}
\end{center}

Data from the BBH merger simulation was provided in pickle (.pkl) format. This was loaded into the program using \verb!df = pickle.load()!. The data column information was obtained with \verb!df.columns!, which were `waveform', `m1', `m2', `merger position', `spin1z', `spin2z', `distance', and `PSNR'. For this analysis, only the `waveform' column was considered with $N$ samples, from which the time array was defined as \verb!np.arange(0, N*dt, dt)! with \verb!dt = 6/N!. The initial dataframe contained 1698 different simulated waveforms, from which, file 178 was chosen at random. This index corresponded to a merge between two black holes, one of mass $102.0 \, M_S$, the other $38.0\, M_S$ with z-spins $0.672$ and $0.612$ respectively, whose peak-signal-to-noise-ratio (PSNR) was $5.726$. The raw waveform was plotted:
\begin{figure}[H]
    \centering
    \includegraphics[width=3.1in]{rawmerger.png}
    \caption*{\textit{Figure 1: Raw BBH Merger signal data, file index 178 of 1697.}}
\end{figure}

\begin{center}
    \textit{B. Filter Design}\\
\end{center}

The next step was to filter the aLIGO noise out for a clearer representation of the signal. This was done by defining a bandpass filter centered at the zero-frequency component to act as a lowpass filter. Two poles were created just off the complex unit circle, whose transfer function is 
\[
    H(z) = M\frac{1}{z-p}  \frac{1}{z-\overline{p}}\tag{8}  
\]
where $p = (1+\epsilon)e^{-2\pi i f_0 dt}$ and $z = e^{-2\pi i f dt}$, with $f_0 = 0$ Hz being the location of the peak. Using (8) and a discrete convolution-based rational feedback filter$^{[4]}$, the denominator was expanded into the input array 
\[
    \verb!den! = \left[1,\, -\frac{2}{1+\epsilon}\cos(2\pi f_0 dt), \, \frac{1}{(1+\epsilon)^2}\right]\tag{9}  
\]
while the numerator array was \verb!num! $= [M]$ with $M = \frac{\epsilon^2}{(1+\epsilon)^2}$ being a theoretically determined normalization by the residue theorem. In this case, the optimal width was experimentally found to be $\epsilon = 0.125$, as this would maximize the amount of noise reduction while simultaneously allowing the chirp frequency to pass through. The power spectrum was then plotted for $f_0 = 0$ Hz with $\epsilon$ by applying a FT algorithm to a delta-function input array \verb![1, 0, 0, ...]!. The result was plotted, including the aliasing frequencies:
\begin{figure}[H]
    \centering
    \includegraphics[width=3.1in]{power spectrum filter.png}
    \caption*{\textit{Figure 2: Power spectrum of designed bandpass filter to filter out aLIGO noise from merger signal. Power spectrum is normalized by $M$. Frequecy axis is accurately scaled and is proportional to the number of input samples of the discretization of the $\delta_n$ array as in the initial signal.}}
\end{figure}
This filter was then applied to the input waveform by convolution$^{[4]}$. The comparison between the unfiltered data and the filtered data was plotted and is shown in Figure 4. The FT of the signal is shown in Figure 4-(B), but is not particularly useful since the frequency is time-dependent. The large spike near $0$ Hz is therefore the base frequency, and the leakage is caused by a positive chirp rate.  

\begin{figure*}[!t]
    \begin{minipage}{0.3\linewidth}
        \includegraphics[width=7in]{filtered merger and fft.png}
    \end{minipage}\rule{0.5em}{0pt}%
    \caption*{\textit{Figure 4: Comparison of the bandpass-filtered merger signal with the initial waveform, showing the significant reduction of aLIGO noise. (A) shows the amplitude-time comparison, while (B) shows the reduction of higher frequences in the FT domain.}}
\nd \hrulefill  
\end{figure*}


\begin{center}
    \textit{C. Isolation of Pre-Merge Data}\\
\end{center}

Since the chirp is only present in the pre-merge data interval, it must be isolated from the post-merge noise in order to perform a LTC on the signal. This was done by directly locating the timestamp where the merge maximum was completed, then determining the sample number by $\frac{T}{6}*N$ where $T$ is the merge completion timestamp. To generalize this operation, \verb!np.argmax()! can be used on the input waveform to determine the index location of the maximum, which would indicate the completion of the merge. The data was then sliced. 

Once the wave tail is removed, a Hanning window was applied to remove spectral leakage and to smoothen the edges of the time series. For the specific chosen waveform, it was found that the pre-merge sample length was approximately 2.5 s, so the Hanning window was defined as 
\[  
    h(t) = \frac{1}{2}(1 - \cos(2\pi t/ 2.5))\tag{10}
\]
from which the filtered, windowed pre-merge waveform was plotted. The windowed and trimmed input is shown in Figure 3.
\begin{figure}[H]
    \centering
    \includegraphics[width=3.4in]{windowed pre-merger.png}
    \caption*{\textit{Figure 3: Plot of the filtered, windowed pre-merge waveform.}}
\end{figure}


\begin{center}
    \textit{D. Linear Chirp Transform}\\
\end{center}

It was first attempted to isolate the frequency-chirp-time dependence by specifying a complex window function as in (6), to later implement a JCTFT function in the form of (7) to extract information about the frequency-time dependence. It was quickly discovered that the determination of these values was not possible due to the density of the data, and this is discussed in the latter sections of this paper. 

Instead, the LCT function, as defined in (2), was implemented into Python in the form of (5). It was programmed by taking the convolution of the FT of both the $f(t)$ input and the $e^{-2\pi i \beta t^2}$ chirp term. The difficulty arose in attempting to FT and convolve multi-dimensional arrays. Although the FT of the input $f(t)$ was a single-dimensional array, the chirp term contained frequency and $\beta$ array inputs. When applying the FT algorithm to the chirp term, it had to be taken along the temporal axis, not the axis specified by $\beta$. Determining this was the first challenge. The next challenge was determining how to convolve both arrays along the same frequency axis since the output had to be chirp-dependent. This was done by using a for-loop, since \verb!np.convolve()! does not take an `axis' argument. Once this function was designed and tested in a secondary Python file, it was implemented on the filtered and windowed BBH merge time signal. The output array was then two-dimensional: one frequency axis, one chirp axis. The maxima along both these axes thus corresponds to the values which compose the input signal.

Using \verb!matplotlib.pyplot.imshow()!, a spectogram of the two-dimensional maximum was generated.

\begin{figure}[H]
    \centering
    \includegraphics[width=3.7in]{chirp spec.png}
    \caption*{\textit{Figure 5: Linear Chirp Transform of the filtered, windowed pre-merge signal plotted in two-dimensions. The darker shades indicate an increase in magnitude, which is shown to intersect just above the center of the image. The spread of color is due to a lack of spectral resolution.}}
\end{figure}



\begin{figure*}[!t]
    \centering
    \hspace{-320pt}
    \begin{minipage}{0.3\linewidth}
        \includegraphics[width=6.7in]{chirp cross section.png}
    \end{minipage}\rule{0.5em}{0pt}%
    \caption*{\textit{Figure 6: Cross-section maxima of the LCT of the filtered, windowed merge signal. In (A), the chirp-axis spectrum, whose spectral resolution is low due to a lack of windowing. In (B), the maximum of the frequency spectrum, whose spectral leakage has been minimized due to windowing of the input time signal.}}
\nd \hrulefill  
\end{figure*}

\begin{spacing}{1.2}

    The location of the maximum was then determined with \verb!np.argmax()!, from which cross-sections of each indexed location was plotted. This plot is shown in Figure 6. It is important to note the differences in chirp rate spectrum and frequency spectrum, since the chirp rate term $e^{-2\pi i \beta t^2}$ was not windowed prior to applying the FT algorithm, hence slight spectral leakage is observed. On the contrary, since the input signal was windowed, spectral leakage was minimized and therefore the base frequency FT of the input is more precise. 

    Lastly, using the maximum information determined by \verb!np.argmax()!, the values of $\omega$ and $\beta$ were taken to plot the frequency evolution of the argument of the form of (1). To compare with the merge waveform, the real component of $e^{-2\pi i (\omega t + \beta t^2)}$ was also plotted on top of the frequency evolution. Note that the complex sinusoid has no dimensions and strictly represents the time evolution of the merge signal. These are shown in Figure 7. For this specific waveform, the values determined from the LCT were $\omega = 7.92\,s^{-1}$ and $\beta = 1.81 \, s^{-2}$. 

\end{spacing}


\begin{figure}[H]
    \centering
    \includegraphics[width=3.1in]{freq evolve.png}
    \caption*{\textit{Figure 7: Time-dependent frequency evolution of the merge signal as determined by the LCT maxima. The red curve shows the frequency-time depdenence. The orange curve shows the rate of change of frequency. The grey curve is the frequency change in a complex sinusoid.}}
\end{figure}




\begin{center}
    \textbf{III \hspace{5pt} DISCUSSION}

\vspace{10pt}

    \textit{A. Results}

\vspace{5pt}
\end{center}

The base frequency $\omega = 7.92 \,s^{-1}$ and chirp rate $\beta = 1.81 \, s^{-2}$ results obtained from the BBH merger signal analysis correspond to an orbiting BBH system whose oscillation frequency is increasing at a rate of $1.81 \text{Hz} / s^{-1}$. These results may reconstruct the input signal as in Figure 7 and can provide information on the processes leading up to the merge$^{[2]}$. One caveat to the analysis is the presence of spectral leakage from the exponential chirp term in the FT, which was not windowed. 



\begin{center}
    \textit{B. Limitations}\\
\end{center}

The primary boundary affecting deeper analysis of the data was the sampling density of the input waveforms. Each 6s waveform contained 12288 samples, which implied that most manipulation to the data would take longer than usual to process, such as any internal-function for-loops, FFT algorithms, or convolutions. Although this wasn't much of a problem for single-dimensional arrays, the introduction of 2- and 3-dimensional arrays made running the program a lot more time consuming. For instance, the FT and filtering was taken out using a \verb!10*dt! sample size to downsample the data, from which was compensated for in the plots by using axis ticks and labels in plots.

\vspace{5pt}

\begin{center}
    \textit{C. Implementation of the JCTFT}\\
\end{center}

\vspace{5pt}

The majority of the project was spent attempting to write a JCTFT function of the form of (7) so that time-frequency dependence could be analyzed directly from the signal, instead of having to interpolate as in the previous section with finding maximum values. This however, introduced a 3-dimensional array, which was very difficult to plot due to the 12288x12288x100 array size including colors. Many attempts were made to merge the chirp data by summing over its axis of nonzero values$^{[2]}$, but even this was far too difficult to test due to the struggles of matching array sizes, applying FT and convolution algorithms with for-loops along specific axes of the array, and by not knowing the non-zero rate axis size. Many functions were written but none were able to be tested and due to the former reasons they were commented out of the program.



\begin{center}
    \textit{D. Future Improvements}\\
\end{center}


To improve the project, it would be beneficial to analyze more waveforms for black holes of various masses and spins to study the effect of spins, masses, and PSNR values on merging signals. This could be taken out using cross-correlation. 

In terms of analysis, the introduction of a time-independent windowing function to the complex exponential chirp term defined in the LCT in equation (5) would be useful for the reduction of spectral leakage from the chirp transform array axis, as shown in Figure 6-(A). This can be done (and was tested) for sinusoids with known values of $\omega$ and $\beta$, but was much more complex for BBH merge signals due to the inconsistencies of noise and amplitude. 
One could lastly consider the isolation of the chirp rate domain by two-dimensional bandpass filtering. This may reduce further spectral leakage along the chirp axis.

\begin{center}
    \textit{E. Conclusion}\\
\end{center}

The chirp rate analysis of BBH merger signals is useful for determining the frequency evolution of the orbiting system in the form of gravitational waves. This was completed using filter design, Hann windowing, modified FT algorithms, and convolution along specific axes. The chosen data sample was a 6s 410 Mpc BBH merger signal between two black holws of mass $102.0\, M_S$ and $38.0 M_S$ with z-spins $0.672$ and $0.612$, respectively, with a PSNR of 5.726. After defining and applying a LCT function in the Python program as defined in (2), it was found that the chirp rate of the merge was $\beta = 1.81\, s^{-2}$ with a base frequency of $\omega = 7.92\, s^{-1}$. These results were plotted. The limitations of the project were primarily dataframe density and difficulties in defining functions. Future improvements of the project were discussed, such as the application of two-dimensional feedback filters to the chirp-frequency domain, and the application of cross-correlation to similar BBH merger signals.





\end{multicols}


\pagebreak 

\textbf{Bibliography}
\begin{enumerate}
    \item[{[1]}] Li, X., 2022. Gravitational Wave signals from Various Sources, V3. Harvard Dataverse. \color{blue}\url{https://dataverse.harvard.edu/dataset.xhtml?persistentId=doi:10.7910/DVN/GEVGRO}\color{black}. Filename: `validation-6s-merger-wnoise410Mpc-waveform\textunderscore data.pkl'. 
    \item[{[2]}] Li, X., Houde, M., Mohanty, J. and Valluri, S.R., 2023. A Joint-Chirp-Rate-Time-Frequency Transform for BBH Merger Gravitational Wave Signal Detection (No. arXiv: 2209.02673). \color{blue}\url{https://arxiv.org/pdf/2209.02673.pdf}\color{black}.
    \item[{[3]}] Alkishriwo, O.A., 2013. The discrete linear chirp transform and its applications (Doctoral dissertation, University of Pittsburgh). \color{blue}\url{https://core.ac.uk/download/pdf/12214635.pdf}\color{black}.
    \item[{[4]}] Alloway, J., 2024. PHY408 Lab 3 Filter Design.
\end{enumerate}
\end{document}